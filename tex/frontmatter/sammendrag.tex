\begin{otherlanguage}{norsk}

\section*{Sammendrag}

Husk at hvis du er en norsk student og skriver masteren din på engelsk, så \textit{må\/} du lage et sammendrag på norsk.
Bruk ikke Google Translate eller lignende, uten skriv teksten direkte på norsk.
Sammendraget trenger absolutt ikke å være identisk ord-for-ord med abstract, men skal selvsagt ha i prinsipp samme innehold, på semantisk nivå.

\end{otherlanguage}

(If you are a non-Norwegian student, it is not obligatory to include an abstract in Norwegian.)

For those who write a Norwegian summary, whatever you do, do \textit{not\/} just directly translate the English abstract.
It might be tempting to think that the Norwegian summary is something you can do on the fly --- maybe assuming that nobody will read it. 
However, in fact the opposite might be true: it is very likely that it will be read by the people you most want to make a good impression on,
such as your friends, family, and future employers. 

