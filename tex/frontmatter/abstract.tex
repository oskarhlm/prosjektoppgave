\section*{Abstract}

\begin{comment}
This paper provides a template for writing a Master's Thesis
(parts of it can also be used when writing a Specialisation Project Report).
The template does not form a compulsory style that you are obliged to use, but rather provides a common starting point for all students. For a given thesis, tuning of the template may still be required, depending on the nature of the thesis and the author's writing style.
Such tuning might involve moving a chapter to a section or vice versa, or removing or adding sections and chapters.

    [If you write a Specialisation Project Report, it should normally focus on the background, related work (i.e., your literature study), and future work sections ---
        with the ``future work'' section containing the plan for the Master's Thesis work to be carried out in the second semester.
        Architectural and experimental sections can also be included, but in preliminary versions.
        All those sections should of course be updated in the Master's Thesis and adapted to the actual work carried out.]

Note that the template contains a lot of examples of how to write different parts of the thesis
as well as how to cite authors and how to use LaTeX and BibTeX.
Some of those examples might only be clear if you actually look at the LaTeX source itself.

The abstract is your sales pitch which encourages people to read your work,
but unlike sales it should be realistic with respect to the contributions of the work.
It should include:
\begin{itemize}
    \item the field of research,
    \item a brief motivation for the work,
    \item what the research topic is,
    \item the research approach(es) applied, and
    \item contributions.
\end{itemize}

The abstract length should be roughly half a page of text (and not more than one page).
It will normally be longer than the abstracts you see in research papers, since some more background / motivation is included.
Do not include lists, tables or figures.
Avoid abbreviations and references.

When writing the abstract, keep in mind that most people might only read this text (and many only the title), so be sure to make it sound good.
What you really want to accomplish is that people who read the abstract will get drawn into your project and read the rest of the text too.
However, the old saying most definitely applies here: You never get a second chance to make a first impression.
\end{comment}

The emergence of powerful \gls{acr:llm} with remarkable reasoning capabilities and coding knowledge---like \acrshort{acr:gpt}-4, the latest additions to the \acrshort{acr:gpt} series---enables automation of a wide range of tasks. ChatGPT's Code Interpreter is able to generate, execute, and review its own code, making creation of autonomous \acrshort{acr:ai} agents far easier than before. This specialization project explores the feasibility of \gls{acr:llm}-based \acrshort{acr:gis} agents, investigating current use cases for ChatGPT in the field of \acrshort{acr:gis}, and how such a system can be implemented. The project seeks to highlight the strengths of current technologies, but also highlight its weaknesses and suggest areas of improvements and possible solutions. These goals are achieved through a literature study and experiments that aim to support the findings of the literature study. The literature study presents the body of work that has already been done in regards to the position of \acrshortpl{acr:llm} in the field of \acrshort{acr:gis}, as well as planning strategies applied in \acrshort{acr:llm}-based agents, and retrieval-augmented generation, that is, giving the \acrshort{acr:llm} \enquote{hooks} into the real world. The three experiments focus on ChatGPT's ability to handle geospatial data in various formats and through different channels.

\glsresetall
