\chapter{Related Work}
\label{cha:related-work}

\begin{comment}
What other research has been conducted in this area and how is it related to your work?
This section is thus where your literature review will be presented. It is important when presenting the review
that you give an overview of the motivating elements of the work going on in your field and how these relate to your work,
rather than a list of contributors and what they have done.
This means that you need to extract the key important factors for your work and discuss how others have addressed
each of these factors and what the advantages/disadvantages are with such approaches.
As you mention other authors, you should reference their work.
Note that the reference list reflects the literature you have read {\em and\/} have cited.
This will only be a subset of the literature that you have read.

A good way to find relevant work is by checking what others are referencing, e.g., in papers you have already found
However, when doing that,
do not fall into one of the common traps, such as re-iterating someone's false quote or faulty analysis of
a previous paper (check the original source!), or getting stuck inside a local research cluster (a group of
researchers that mainly refer to the ones using the same type of approaches or similar ideas).

Make sure that it is clear how and why you decided to include some references (and discard others). As in all parts of research, it should ideally be possible for someone else to reproduce your work, also when it comes to finding the relevant references.
There are (at least) three basic methods for finding references:
\begin{enumerate}
    \item Trust the authorities (e.g., your supervisor) to dig out good texts for you.
          Those can often be used as a seed set for:
    \item Snowballing, where you have some good articles and check the references in them for other good ones.
          Note that this can be done both backwards and forwards on the timeline; that is, using tools like Google Scholar, you can also check who refers \textit{to\/} the good articles you have already found.
\end{enumerate}

Note that a reference needs to be complete: you should always give the full name of a conference or journal,
always include page numbers, always say where a book or thesis was published, and where a conference took place, as further described in Section~\ref{sec:reference_list}.
\end{comment}

\cite{liAutonomousGISNextgeneration2023} states that “autonomous GIS will need to achieve five autonomous goals: self-generating, self-organizing, self-verifying, self-executing, and self-growing.”, and provide a “divide-and-conquer”-based method to address some of these goals. Furthermore, they propose a simple trial-and-error approach to addressing the self-verifying goal. They also highlight the inherit need of a memory system in a mature LLM-based GIS system, referring to the use of vector databases in autonomous agents like AutoGPT \citep{richardAutoGPTHeartOpensource2023}. Even with its shortages, the solution that \citep{liAutonomousGISNextgeneration2023} provide, called LLM-Geo, is able to solve provide good solutions in various case studies by providing executable assemblies in a Python environment when provided with URLs to relevant data sets, along with a user-specified query.

\cite{zhangGeoGPTUnderstandingProcessing2023} uses the Langchain framework \citep{chaseLangChain2022} in order to combine different GIS tools in a sequence in order to solve different sub-goals, and focuses on using the semantic understanding and reasoning abilities of LLMs like (e.g., ChatGPT) to call externally defined tools, employing the LLM as an agent or controller. The authors take great inspiration from the AutoGPT framework \citep{richardAutoGPTHeartOpensource2023}. The externally defined tools are described (manually) by its name and description. Said description contains information about the input parameters and output types of the tools/functions. Tools are defined for geospatial data collection, data processing and analysis, and data visualization. The effectiveness of the system is showcased in four case studies.

\glsresetall