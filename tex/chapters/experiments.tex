\chapter{Experiments and Results}
\label{cha:experiments}

\section{Experimental Plan}
\label{sec:experimentalPlan}

\begin{comment}
Trying and failing is a major part of research. However, to have a chance of success you need a plan driving the experimental research, just as you need a plan for your literature search. Further, plans are made to be revised, and this revision ensures that any further decisions made are in line with the work already completed.

The plan should include what experiments or series of experiments are planned and what questions the individual or set of experiments aim to answer. Such questions should be connected to your research questions, so that in the evaluation of your results you can discuss the results wrt to the research questions.
\end{comment}

The experiments conducted for this report have been planned to answer the research questions described in \autoref{sec:goals-and-research-questions}.

\subsection[Determining the Potential of LLM-based GIS analysis (RQ 1)]{Determining the Potential of LLM-based GIS analysis (\rqref{rq:llm-potential})}

\rqref{rq:llm-potential} differs from \rqref{rq:overlay-analysis} and \rqref{rq:external-tools} in that it is more open-ended. The tests will try to display the abilities of \acrshort{acr:gpt}-4 for geospatial tasks, without providing it with any context or external tools. The tests are inspired by the work of \cite{robertsGPT4GEOHowLanguage2023} (see \autoref{subsec:gis-with-llms}), who did experiments with increasing difficult on \acrshort{acr:gpt}-4 to characterize what \acrshort{acr:gpt}-4 knows about the geographical world, highlighting both capabilities and limitations.

\subsection[Testing ChatGPT's Ability to Perform Overlay Analysis using OGC API - Features (RQ 2)]{Testing ChatGPT's Ability to Perform Overlay Analysis using \acrshort{acr:ogc} \acrshort{acr:api} - Features (\rqref{rq:overlay-analysis})}

Experiments on \rqref{rq:overlay-analysis} will require three different elements:

\begin{enumerate}
    \item Input data in correspondence with the \acrshort{acr:ogc} \acrshort{acr:api} - Features specification
    \item Prompts to ChatGPT-4 using the "Code Interpreter" beta feature
    \item Gold standard/expected output for the given combination of input data and prompts
\end{enumerate}

\subsection[Testing ChatGPT's Ability to To Use External Tools (RQ 2)]{Testing ChatGPT's Ability to To Use External Tools (\rqref{rq:external-tools})}

Experiments on \rqref{rq:external-tools} will test different tools and techniques intended to give an \acrshort{acr:llm} access to up-to-date information and external tools (see \autoref{subsec:retrieval-automented-generation} on \acrlong{acr:rag}). The main focus will be on how one can use a tool like \hyperref[subsubsec:langchain]{LangChain} hook up a conversational \acrshort{acr:ai} with sophisticated functionality found in various \acrshort{acr:gis} software.

\section{Experimental Setup}
\label{sec:experimentalSetup}

\begin{comment}
The experimental setup should include all data --- parameters, etc. --- that would allow a person to repeat your experiments.
This will thus be the actual instantiation for each experiment of the general architecture described in Chapter~\ref{cha:architecture}.
\end{comment}

\section{Experimental Results}
\label{sec:experimentalResults}

\begin{comment}
Results should be clearly displayed and should provide a suitable representation of your results for the points you wish to make.
Graphs should be labelled in a legible font. If more than one result is displayed in the same graph, then these should be clearly marked.
Please choose carefully rather than presenting every result. Too much information is hard to read and often hides the key information you wish to present. Make use of statistical methods when presenting results, where possible to strengthen the results.
Further, the format of the presentation of results should be chosen based on what issues in the results you wish to highlight.
You may wish to present a subset in the experimental section and provide additional results in an appendix.
Point out specifics here but save the overall/general discussion to the Discussion chapter.
\end{comment}

\glsresetall
