\chapter{Experiments and Results}
\label{cha:experiments}

\section{Experimental Plan and Setup}
\label{sec:experimental-plan-and-setup}

\begin{comment}
Trying and failing is a major part of research. However, to have a chance of success you need a plan driving the experimental research, just as you need a plan for your literature search. Further, plans are made to be revised, and this revision ensures that any further decisions made are in line with the work already completed.

The plan should include what experiments or series of experiments are planned and what questions the individual or set of experiments aim to answer. Such questions should be connected to your research questions, so that in the evaluation of your results you can discuss the results wrt to the research questions.
\end{comment}

The experiments conducted for this report have been planned to answer the research questions described in \autoref{sec:goals-and-research-questions}.

\subsection[RQ1: Determining the Potential of LLM-based GIS analysis]{\rqref{rq:llm-potential}: Determining the Potential of LLM-based GIS analysis}

\rqref{rq:llm-potential} differs from \rqref{rq:overlay-analysis} and \rqref{rq:external-tools} in that it is more open-ended. The tests will try to display what abilities \acrshort{acr:gpt}-4 has geospatial tasks out of the box, without providing it with any context or external tools.

\subsubsection{Ability to Conduct Geospatial Anaylsis}

Tests were performed to assess ChatGPT's ability to perform geospatial analysis. The testing approach is inspired by the work of \cite{robertsGPT4GEOHowLanguage2023} (see \autoref{subsec:gis-with-llms}), who did experiments with increasing difficult on \acrshort{acr:gpt}-4 to characterize what \acrshort{acr:gpt}-4 knows about the geographical world, highlighting both capabilities and limitations. These focused on \acrshort{acr:gpt}-4's general geospatial awareness, and were not concerned with \acrshort{acr:gis}-related tasks. Therefore, I will refer to \cite{robertsGPT4GEOHowLanguage2023} when highlighting its somewhat surprising geospatial awareness abilities, and focus my efforts to displaying its potential for use in the world of \acrshort{acr:gis}. I will do this by constructing various tests that try to reflect its GIS knowledge.

I will use the Elveg 2.0 dataset \citep{thenorwegianmappingauthorityElveg2019}, along with cadastral and land use data. In order to assess ChatGPT's ability to read and understand different data formats, the data will be provided in \acrshort{acr:sosi}, \acrshort{acr:gml}, and shapefile formats. \autoref{enum:gpt-gis-questions} lists the questions that were asked, in rising order of complexity.

\begin{enumerate}
    \item \enquote{Provide a summary of the datasets, highlighting its most salient features.}
    \item \enquote{Extract all roads with a speed limit less than or equal to 60 km/h.}
    \item \enquote{Select all buildings located within 50 metes of a highway.}
    \item \enquote{Find the area best suited for expansion to accommodate residential buildings.}
    \item \enquote{Assess the environmental impact of the proposed expansion.}
\end{enumerate}
\label{enum:gpt-gis-questions}

\subsubsection{Data Access}

Another important thing to test is the issue of providing ChatGPT with relevant files on which it can perform analyses. ChatGPT Plus users will have access a range of advanced features, including web browsing with Bing, Dall-E Image Generation, and Code Interpreter. The latter of these allows the user to manually upload files into the chat instance and perform advanced analyses on the contents of these. While this is very powerful, having to manually upload files poses some limitations. A more flexible system should be capable of accessing web \acrshortpl{acr:api} in real time.

A dataset containing the border of Drammen Municipality was used to test compare ChatGPT's ability to perform analyses on manually uploaded data, versus data handed through it from passing a URL address. The data conforms to the GeoJSON standard and contains a FeatureCollection object with a single Feature, namely the border.

In order to test this, a dataset  how ChatGPT Code Interpreter can handle different file types, and whether it can access data through web \acrshortpl{acr:api} efficiently.

\subsection[RQ2: Testing ChatGPT's Ability to Perform Overlay Analysis using OGC API - Features]{\rqref{rq:overlay-analysis}: Testing ChatGPT's Ability to Perform Overlay Analysis using \acrshort{acr:ogc} \acrshort{acr:api} - Features}

Experiments on \rqref{rq:overlay-analysis} will require three different elements:

\begin{enumerate}
    \item Provide ChatGPT with \acrshort{acr:api} URL(s) to relevant data collections corresponding with the \acrshort{acr:ogc} \acrshort{acr:api} - Features specification
    \item Prompts to ChatGPT-4 using the "Code Interpreter" beta feature
    \item Gold standard/expected output for the given combination of input data and prompts
\end{enumerate}

These experiments will be limited to the collections found at \url{https://alenos-tester001.azurewebsites.net/}. This example \acrshort{acr:ogc} \acrshort{acr:api} is created by Norkart's Alexander Salveson Nossum for with the purpose of testing \acrshort{acr:ogc} \acrshort{acr:api} Features on Norwegian data. It was created using \texttt{pygeoapi}\footnote{\url{https://pygeoapi.io/}}, which is a Python server implementation of the \acrshort{acr:ogc} \acrshort{acr:api} suite of standards. It allows for deployment of a RESTful \acrshort{acr:ogc} \acrshort{acr:api} endpoint using OpenAPI, GeoJSON, and HTML.

\subsection[RQ3: Testing ChatGPT's Ability to To Use External Tools]{\rqref{rq:external-tools}: Testing ChatGPT's Ability to Use External Tools}

Experiments on \rqref{rq:external-tools} will test different tools and techniques intended to give an \acrshort{acr:llm} access to up-to-date information and external tools (see \autoref{subsec:retrieval-automented-generation} on \acrlong{acr:rag}). The main focus will be on how one can use a tool like \hyperref[subsubsec:langchain]{LangChain} hook up a conversational \acrshort{acr:ai} with sophisticated functionality found in various \acrshort{acr:gis} software.

\section{Experimental Results}
\label{sec:experimentalResults}

\subsection[RQ1: Determining the Potential of LLM-based GIS analysis]{\rqref{rq:llm-potential}: Determining the Potential of LLM-based GIS analysis}

\subsection{Data Access}




\begin{comment}
Results should be clearly displayed and should provide a suitable representation of your results for the points you wish to make.
Graphs should be labelled in a legible font. If more than one result is displayed in the same graph, then these should be clearly marked.
Please choose carefully rather than presenting every result. Too much information is hard to read and often hides the key information you wish to present. Make use of statistical methods when presenting results, where possible to strengthen the results.
Further, the format of the presentation of results should be chosen based on what issues in the results you wish to highlight.
You may wish to present a subset in the experimental section and provide additional results in an appendix.
Point out specifics here but save the overall/general discussion to the Discussion chapter.
\end{comment}

\glsresetall
