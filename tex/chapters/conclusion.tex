\chapter{Conclusion and Future Work}
\label{cha:conclusion}

\begin{comment}
What are the main contributions made to the field?
How significant are these contributions?
Also discuss the contributions in terms of the goals and research questions formulated in the Introduction.

The contributions section will normally contain everything that you address in the abstract, but in an extended form and quite possibly additional issues that cannot be included in the abstract.
An obvious difference is that when the reader has come this far in the text, she/he should be quite familiar with the work, but while reading the abstract they will have little to no knowledge of the work.

The section ``Contributions'' in Chapter~\ref{cha:introduction} differs from this one in that the former is just a list of the main bits, while this section should explain them in more detail.
However, basically the same items should appear in both sections.

\section{Future Work}
\label{sec:futureWork}

Consider where you would like to extend or improve this work, or how somebody else could continue it.
These extensions might either be continuing the ongoing direction or taking a side direction that became obvious during the work.
Further, possible solutions to limitations in the work conducted, highlighted in Section~\ref{sec:discussion} may be presented.

Note that in the Specialisation Project Report, the Future Work section will be a key part of your plan for the novel work to be carried out in the next semester,
while in the Master's Thesis, the Future Work section rather will point to issues that others might be interested in addressing.
This can include options and alternatives that you did not try out yourself, or potential improvements and extensions to your experiments or system.
\end{comment}

\section{Conclusion}
\label{sec:conclusion}



\section{Future Work}
\label{sec:futureWork}

Seeing as this is a specialization project that will transition into a larger master thesis, the points discussed in this section will be key parts of the plan for the work carried out in the latter.

\subsection{Test regime}

In order to test the feasibility of different language models to serve as the brain of an autonomous \acrshort{acr:gis} agent, a testing regime should be developed. In the examples of autonomous \acrshort{acr:gis} agents described in the literature study of this report (see \autoref{sec:gis-with-llms}), results have generally been presented in the form of case studies \citep{liAutonomousGISNextgeneration2023,zhangGeoGPTUnderstandingProcessing2023}. This type of qualitative testing is entirely appropriate to showcase the possibilities of the technologies but may be insufficient when comparing performance of different systems. In the latter case a quantitative approach would probably be preferable.

One idea is to create a test dataset which consists of inputs and corresponding desired outputs of typical \acrshort{acr:gis} tasks. Inputs would in this case be natural language queries inputted by a mock user, and the output would be what you would expect a \acrshort{acr:gis} professional to return when given the same tasks/queries. Inputs should reflect the varying level of \acrshort{acr:gis} knowledge in the different user groups (see \autoref{sec:user-groups}). Outputs could be files with typical geospatial extensions (.shp, .geojson, .sosi, etc.), or they could adhere to API schemas specified by geospatial standards (see \autoref{sec:geospatial-standards}).

While the inputs should be fairly simple to construct there are several questions to be answered in regard to the outputs:

\begin{itemize}
    \item How does one evaluate the accuracy of the output?
    \item How should the \acrshort{acr:ai} agent respond when the user does not specify an output file format?
    \item How does one evaluate the usefulness of outputs to questions that should not return geospatial files, e.g. answers to general questions about geo-related subjects?
\end{itemize}

These are questions outside the scope of this specialization project. They will, however, be pursued in my master thesis.

\subsection{Framework for Planning, Acting, and Reasoning}

\subsection{Embeddings}

\subsection{Fine-Tuning}



