\chapter{Conclusion and Future Work}
\label{cha:conclusion}

\textit{Lorem ipsum dolor sit amet, consectetur adipiscing elit. Nam consequat pulvinar hendrerit. Praesent sit amet elementum ipsum. Praesent id suscipit est. Maecenas gravida pretium magna non interdum. Donec augue felis, rhoncus quis laoreet sed, gravida nec nisi. Fusce iaculis fermentum elit in suscipit. }

\section{Contributions}
\label{sec:contributions}

What are the main contributions made to the field?
How significant are these contributions?
Also discuss the contributions in terms of the goals and research questions formulated in the Introduction. 

The contributions section will normally contain everything that you address in the abstract, but in an extended form and quite possibly additional issues that cannot be included in the abstract. 
An obvious difference is that when the reader has come this far in the text, she/he should be quite familiar with the work, but while reading the abstract they will have little to no knowledge of the work.

The section ``Contributions'' in Chapter~\ref{cha:introduction} differs from this one in that the former is just a list of the main bits, while this section should explain them in more detail.
However, basically the same items should appear in both sections.

\section{Future Work}
\label{sec:futureWork}

Consider where you would like to extend or improve this work, or how somebody else could continue it.
These extensions might either be continuing the ongoing direction or taking a side direction that became obvious during the work. 
Further, possible solutions to limitations in the work conducted, highlighted in Section~\ref{sec:discussion} may be presented. 

Note that in the Specialisation Project Report, the Future Work section will be a key part of your plan for the novel work to be carried out in the next semester,
while in the Master's Thesis, the Future Work section rather will point to issues that others might be interested in addressing.
This can include options and alternatives that you did not try out yourself, or potential improvements and extensions to your experiments or system.