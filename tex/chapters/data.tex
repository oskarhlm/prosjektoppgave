\chapter{Datasets}
% OR: \chapter{Data}
\label{cha:data}

\begin{comment}
You will (probably) need to describe and discuss the dataset(s) that you use in your work.
Depending on how much detail is needed and whether you have done any work on the data yourself
(including analysing it, collecting or annotating some of it, or cleaning/preprocessing it),
the data description can possibly be part of the Background chapter, the Related Work chapter,
the Architecture chapter or the Experimental Setup.
The dataset(s) can also be described in a separate chapter, either before or after the chapter on related work.
Note that if you have put some effort of your own into the data, you will need to make sure that the text about it
is part of the ``foreground'' (your own work) rather than the ``background'' (everything done by somebody else), which includes the theoretical
background chapter(s) and the related work.

\begin{table}[t!]
    \centering
    \begin{tabular}{l|cccc|c}
        \tabletop
        Dataset   & Normal & Offensive & Hateful & Spam    & Total           \\
        \tablemid
        Original  & 53,790 & 27,037    & 4,948   & 14,024  & \textbf{99,799} \\
        Available & 41,784 & 14,202    & 2,941   & ~~9,372 & \textbf{68,299} \\
        \tablebot
    \end{tabular}
\end{table}
\end{comment}

\glsresetall
